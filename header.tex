\documentclass[
	table,
  numbers=noenddot,			% Kein . am Ende einer Nummerierung
  bibliography=totoc,       % Literatur im Inhaltsverzeichnis
  listof=totoc,				% Verzeichnisse im Inhaltsverzeichnis
  captions=tableheading,    % Tabellenüberschriften
]{scrbook}

\usepackage[table]{xcolor}% http://ctan.org/pkg/xcolor
\usepackage[automark]{scrlayer-scrpage}
\setcounter{tocdepth}{4}   %Maximale Gliederungsebene im Inhaltsverziechnis
\setcounter{secnumdepth}{4}   %Maximale Nummerierungsebene im Inhaltsverziechnis
\KOMAoptions{twoside=false}	 	%Einseitige einstellung

% Bei der Klasse "book" ist LaTeX bemüht, den Text auf zwei gegenüberliegenden Seiten auf die gleiche Höhe zu bringen.
% Dies wird dadurch erreicht, dass der Text unter Überschriften oder Zeilenumbrüchen minimal (und nicht sichtbar) variiert wird.
% Bei manchen Dokumenten funktioniert das nicht, da - beispielsweise durch viele Grafiken und wenig Text - nicht genug Spielraum bleibt.
% Das kann dazu führen, dass der Text in einem bestimmten Abstand vom unteren Seitenrand endet und oben noch sehr viel frei ist.
% Durch Aktivieren oder Kommentieren der folgenden Zeilen, kann dieses Verhalten beeinflusst werden.

%%%%%Checken welches besser ist!!!
\flushbottom  	%Der Text aller Seiten ist gleich lang
%\raggedbottom 	%Der Text kann auch unterschiedlich lang sein.

%Abschnittsnummerierungen in Caption und Formelnummerierung
\usepackage{chngcntr}

%Definition der Seitenränder
\usepackage[
  left=4cm,	%Linker Seitenrand (Bindung beachten!!!)
  right=2cm,
  top=3.5cm,
  bottom=3cm
]{geometry}

% Paket float verbessern
\usepackage{scrhack}

%Einstellbarer Zeilenabstand
\usepackage{setspace}

% Warnung, falls nochmal kompiliert werden muss
\usepackage[aux]{rerunfilecheck}

% unverzichtbare Mathe-Befehle
\usepackage[fleqn]{amsmath}
\setlength\mathindent{1cm}%Abstand der Formeln vom linken Seitenrand

% viele Mathe-Symbole
\usepackage{amssymb}

% Erweiterungen für amsmath
\usepackage{mathtools}

% Fonteinstellungen
\usepackage{fontspec}
% Latin Modern Fonts werden automatisch geladen
% Alternativ:
%\setromanfont{Libertinus Serif}
%\setsansfont{Libertinus Sans}
%\setmonofont{Libertinus Mono}
%\recalctypearea % Wenn man andere Schriftarten gesetzt hat,
% sollte man das Seiten-Layout neu berechnen lassen

% deutsche Spracheinstellungen
\usepackage{polyglossia}
\setmainlanguage{german}

\usepackage[
  math-style=ISO,    % ┐
  bold-style=ISO,    % │
  sans-style=italic, % │ ISO-Standard folgen
  nabla=upright,     % │
  partial=upright,   % ┘
  warnings-off={           % ┐
    mathtools-colon,       % │ unnötige Warnungen ausschalten
    mathtools-overbracket, % │
  },                       % ┘
]{unicode-math}

% traditionelle Fonts für Mathematik
\setmathfont{Latin Modern Math}
% Alternativ:
%\setmathfont{Libertinus Math}

\usepackage{amsmath}
\usepackage{amssymb}\setmathfont{XITS Math}[range={scr, bfscr}]

\setmathfont{XITS Math}[range={scr, bfscr}]
\setmathfont{XITS Math}[range={cal, bfcal}, StylisticSet=1]
\setlength{\delimitershortfall}{-1sp}

%\setmathfont{Cambria Math}

% Zahlen und Einheiten
\usepackage[
  locale=DE,                   % deutsche Einstellungen
  separate-uncertainty=true,   % immer Fehler mit \pm
  per-mode=symbol-or-fraction, % / in inline math, fraction in display math
  range-phrase = ~\text{bis}~ ,
]{siunitx}

% richtige Anführungszeichen
\usepackage[autostyle]{csquotes}

% schöne Brüche im Text
\usepackage{xfrac}

% Standardplatzierung für Floats einstellen
% Die Reihenfolge der Befehle bestimmt was Priorität hat
% tb!hp bedeutet versuche eine Grafik oder Tabelle zuerst oben auf der Seite zu plazieren
% wenn das nicht geht versuche sie unten auf der Seite zu platzieren
% wenn das auch nicht geht setze sie genau an die Stelle wo sie im Text vorkommt
% und im Worst-Case schiebe sie auf eine separate Seite
\usepackage{float}
\floatplacement{figure}{tb!hp}
\floatplacement{table}{tb!hp}

% Floats innerhalb einer Section halten
\usepackage[
  section, % Floats innerhalb der Section halten
  below,   % unterhalb der Section aber auf der selben Seite ist ok
]{placeins}

% Seite drehen für breite Tabellen: landscape Umgebung
\usepackage{pdflscape}

% Captions schöner machen.
\usepackage[
  labelfont=bf,        % Tabelle x: Abbildung y: ist jetzt fett
  font=small,          % Schrift etwas kleiner als Dokument
  width=0.9\textwidth, % maximale Breite einer Caption schmaler
  labelsep=colon,      % Doppelpunkt als Trenner
  format=plain,%Beschriftung wird als Absatz gesetzt
  indention=0pt%Einzug der Beschriftung ist 0
]{caption}

% subfigure, subtable, subref
\usepackage{subcaption}

% Grafiken können eingebunden werden
\usepackage{graphicx}

%Erstellen von Plots
\usepackage{pgfplots}
\pgfplotsset{compat=1.14}
\usetikzlibrary{calc}

% Schaltpläne
\usepackage[european,siunitx,straightvoltages]{circuitikz}

% schöne Tabellen
\usepackage{booktabs}

% Verbesserungen am Schriftbild
\usepackage{microtype}

% Literaturverzeichnis
\usepackage[
  backend=biber,
  isbn=false,
  style=numeric
]{biblatex}

% Quellendatenbank
\addbibresource{bibliography.bib}

%Mehr Einstellungsmöglichkeiten bei Aufzählungen
\usepackage{enumitem}

%Für Einbinden von z.B. pdf_tex mit relativem Pfad
\usepackage{import}

%Konfiguration der Todos
\usepackage[prependcaption]{todonotes}			%Für Todos eingeblendet
%\usepackage[prependcaption,disable]{todonotes}	%Für Todos ausgeblendet

%Für Programmlistings
\usepackage{listings}

%Für MATLAB Listings
\usepackage[numbered,framed]{matlab-prettifier}

%Setzen von Optionen zur Darstellung von Listings
%Quelle: https://en.wikibooks.org/wiki/LaTeX/Source_Code_Listings (angepasst)
\lstdefinestyle{C}{
	backgroundcolor=\color{white},     % choose the background color; you must add \usepackage{color} or \usepackage{xcolor}; should come as last argument
	basicstyle=\footnotesize\ttfamily, % the size of the fonts that are used for the code
	breakatwhitespace=false,           % sets if automatic breaks should only happen at whitespace
	breaklines=true,                   % sets automatic line breaking
	captionpos=b,                      % sets the caption-position to bottom
	commentstyle=\itshape\color{purple!40!black},    % comment style
	deletekeywords={...},              % if you want to delete keywords from the given language
	escapeinside={\%*}{*)},            % if you want to add LaTeX within your code
	extendedchars=true,                % lets you use non-ASCII characters; for 8-bits encodings only, does not work with UTF-8
	frame=single,	                   % adds a frame around the code
	keepspaces=true,                   % keeps spaces in text, useful for keeping indentation of code (possibly needs columns=flexible)
	keywordstyle=\bfseries\color{green!40!black},       % keyword style
	language=C,                 	   % the language of the code
	morekeywords={*,...},              % if you want to add more keywords to the set
	numbers=left,                      % where to put the line-numbers; possible values are (none, left, right)
	numbersep=5pt,                     % how far the line-numbers are from the code
	numberstyle=\tiny\color{gray},     % the style that is used for the line-numbers
	rulecolor=\color{black},           % if not set, the frame-color may be changed on line-breaks within not-black text (e.g. comments (green here))
	showspaces=false,                  % show spaces everywhere adding particular underscores; it overrides 'showstringspaces'
	showstringspaces=false,            % underline spaces within strings only
	showtabs=false,                    % show tabs within strings adding particular underscores
	stepnumber=1,                      % the step between two line-numbers. If it's 1, each line will be numbered
	stringstyle=\color{orange},        % string literal style
	identifierstyle=\color{blue},
	tabsize=2,	                   	  % sets default tabsize to 2 spaces
	title=\lstname                    % show the filename of files included with \lstinputlisting; also try caption instead of title
}


\lstdefinestyle{C++}{
	backgroundcolor=\color{white},   % choose the background color; you must add \usepackage{color} or \usepackage{xcolor}; should come as last argument
	basicstyle=\footnotesize\ttfamily,        % the size of the fonts that are used for the code
	breakatwhitespace=false,         % sets if automatic breaks should only happen at whitespace
	breaklines=true,                 % sets automatic line breaking
	captionpos=b,                    % sets the caption-position to bottom
	commentstyle=\itshape\color{purple!40!black},    % comment style
	deletekeywords={...},            % if you want to delete keywords from the given language
	escapeinside={\%*}{*)},          % if you want to add LaTeX within your code
	extendedchars=true,              % lets you use non-ASCII characters; for 8-bits encodings only, does not work with UTF-8
	frame=single,	                   % adds a frame around the code
	keepspaces=true,                 % keeps spaces in text, useful for keeping indentation of code (possibly needs columns=flexible)
	keywordstyle=\bfseries\color{green!40!black},       % keyword style
	language=C++,                 	   % the language of the code
	morekeywords={*,...},            % if you want to add more keywords to the set
	numbers=left,                    % where to put the line-numbers; possible values are (none, left, right)
	numbersep=5pt,                   % how far the line-numbers are from the code
	numberstyle=\tiny\color{gray},   % the style that is used for the line-numbers
	rulecolor=\color{black},         % if not set, the frame-color may be changed on line-breaks within not-black text (e.g. comments (green here))
	showspaces=false,                % show spaces everywhere adding particular underscores; it overrides 'showstringspaces'
	showstringspaces=false,          % underline spaces within strings only
	showtabs=false,                  % show tabs within strings adding particular underscores
	stepnumber=1,                    % the step between two line-numbers. If it's 1, each line will be numbered
	stringstyle=\color{orange},      % string literal style
	identifierstyle=\color{blue},
	tabsize=2,	                     % sets default tabsize to 2 spaces
	title=\lstname                   % show the filename of files included with \lstinputlisting; also try caption instead of title
}


\lstdefinestyle{Java}{
	backgroundcolor=\color{white},   % choose the background color; you must add \usepackage{color} or \usepackage{xcolor}; should come as last argument
	basicstyle=\footnotesize\ttfamily,        % the size of the fonts that are used for the code
	breakatwhitespace=false,         % sets if automatic breaks should only happen at whitespace
	breaklines=true,                 % sets automatic line breaking
	captionpos=b,                    % sets the caption-position to bottom
	commentstyle=\itshape\color{purple!40!black},    % comment style
	deletekeywords={...},            % if you want to delete keywords from the given language
	escapeinside={\%*}{*)},          % if you want to add LaTeX within your code
	extendedchars=true,              % lets you use non-ASCII characters; for 8-bits encodings only, does not work with UTF-8
	frame=single,	                   % adds a frame around the code
	keepspaces=true,                 % keeps spaces in text, useful for keeping indentation of code (possibly needs columns=flexible)
	keywordstyle=\bfseries\color{green!40!black},       % keyword style
	language=Java,                 	   % the language of the code
	morekeywords={*,...},            % if you want to add more keywords to the set
	numbers=left,                    % where to put the line-numbers; possible values are (none, left, right)
	numbersep=5pt,                   % how far the line-numbers are from the code
	numberstyle=\tiny\color{gray},   % the style that is used for the line-numbers
	rulecolor=\color{black},         % if not set, the frame-color may be changed on line-breaks within not-black text (e.g. comments (green here))
	showspaces=false,                % show spaces everywhere adding particular underscores; it overrides 'showstringspaces'
	showstringspaces=false,          % underline spaces within strings only
	showtabs=false,                  % show tabs within strings adding particular underscores
	stepnumber=1,                    % the step between two line-numbers. If it's 1, each line will be numbered
	stringstyle=\color{orange},      % string literal style
	identifierstyle=\color{blue},
	tabsize=2,	                     % sets default tabsize to 2 spaces
	title=\lstname                   % show the filename of files included with \lstinputlisting; also try caption instead of title
}


\lstdefinestyle{ML}{
	basicstyle         = \mlttfamily,
	language           = Matlab,
	style              = Matlab-editor,
	escapechar         = `,
	mlshowsectionrules = true,
	captionpos=b,
}

% Hyperlinks im Dokument
\usepackage[
 hidelinks,       %Keine Markierung um den Link
 unicode,        % Unicode in PDF-Attributen erlauben
 pdfusetitle,    % Titel, Autoren und Datum als PDF-Attribute
 pdfcreator={},  % ┐ PDF-Attribute säubern
 pdfproducer={}, % ┘
]{hyperref}

%Abkürzungsverzeichnis
\usepackage[nopostdot,style=super,nonumberlist,toc]{glossaries}
%Anpassen der Breite der zweiten Spalte um einen zu frühen Zeilenumbruch zu verhindern
\setlength{\glsdescwidth}{0.8\textwidth}
\newglossary[slg]{symbols}{sym}{sbl}{Symbolverzeichnis}%Definition Symbolverzeichnis
\loadglsentries{glossary.tex}%Einbinden de Glossars
\makeglossaries

% erweiterte Bookmarks im PDF
\usepackage{bookmark}

% Trennung von Wörtern mit Strichen
\usepackage[shortcuts]{extdash}

%Kopf/Fußzeile=====================
\KOMAoptions{headsepline=true}
\KOMAoptions{footsepline=true}
%\rohead*{\textnormal{\headmark}}
%\lehead*{\textnormal{\headmark}}
\cofoot*{\pagemark}
\cefoot*{\pagemark}
\rofoot*{}
\lefoot*{}

\newpairofpagestyles{kapitel}{%Kapitelseiten formatieren
	\clearpairofpagestyles
	\KOMAoptions{headsepline=false}
	\KOMAoptions{footsepline=true}
	%\lofoot*{\footnotesize\textnormal{}}
	\cofoot*{\pagemark}
	\cefoot*{\pagemark}
}
\renewcommand*{\chapterpagestyle}{kapitel}
%Kopf/Fußzeile=====================

%Anpassen der Abbildungs- und Tabellennummerierung========
\counterwithin*{figure}{chapter}
\counterwithin*{figure}{section}
\counterwithin*{figure}{subsection}
\counterwithin*{figure}{subsubsection}

\renewcommand{\thefigure}{%
 \ifnum\value{section}=0
  \thechapter-\arabic{figure}%
 \else
  \ifnum\value{subsection}=0
	\thesection-\arabic{figure}%
  \else
   \ifnum\value{subsubsection}=0
	\thesubsection-\arabic{figure}%
   \else
	 \thesubsubsection-\arabic{figure}%
   \fi
  \fi
 \fi
}

\counterwithin*{table}{chapter}
\counterwithin*{table}{section}
\counterwithin*{table}{subsection}
\counterwithin*{table}{subsubsection}

\renewcommand{\thetable}{%
 \ifnum\value{section}=0
  \thechapter-\arabic{table}%
 \else
  \ifnum\value{subsection}=0
   \thesection-\arabic{table}%
  \else
   \ifnum\value{subsubsection}=0
 \thesubsection-\arabic{table}%
   \else
	\thesubsubsection-\arabic{table}%
   \fi
  \fi
\fi
}

\counterwithin*{equation}{chapter}
\counterwithin*{equation}{section}
\counterwithin*{equation}{subsection}
\counterwithin*{equation}{subsubsection}

\renewcommand{\theequation}{%
 \ifnum\value{section}=0
  \thechapter-\arabic{equation}%
 \else
  \ifnum\value{subsection}=0
   \thesection-\arabic{equation}%
  \else
   \ifnum\value{subsubsection}=0
	\thesubsection-\arabic{equation}%
   \else
    \thesubsubsection-\arabic{equation}%
   \fi
  \fi
 \fi
}

%Da die Bezeichnung nun länger ist, muss der Abstand in den Verzeichnissen angepasst werden
\makeatletter
\renewcommand*\l@figure{\@dottedtocline{1}{1.5em}{4.2em}}% 3.2em statt 2.3em
\let\l@table\l@figure
\makeatother
%Ende Anpassen der Abbildungs- und Tabellennummerierung========

%Komplexe Zahlen
\newcommand{\komp}[2]{\underline{#1}_\text{#2}}

%Formelzeichen
\newcommand{\flet}[2]{#1_\text{#2}}

%Gleiche Linienstärke in circutikz für Leitungen und Bipoles
\ctikzset{bipoles/thickness=1}

%Einheit für Jahr
\DeclareSIUnit[number-unit-product = \,]{\year}{yr}

% Werden Funktionen in LaTeX direkt geplottet, kann dies zu einer sehr langen Kompilezeit führen.
% Aus diesem Grund wird die Anzahl der zu berechnenden Datenpunkte in einer Variable definiert und kann
% so bei Bedarf reduziert, und für die Endfassung entsprechend erhöht werden.
\newcommand{\NumOfSampels}{1000}%Anzahl der Samples für Plots

