\chapter{Kurzfassung}
Um in Zukunft unabhängiger von Importen zu sein und die Energieversorgung nachhaltiger zu gestalten, hat die Bundesregierung das Ziel für die Erzeugung von grünem Wasserstoff durch Elektrolyse von fünf auf zehn Gigawatt Leistung im Jahr 2030 angehoben. Davon sollen drei Gigawatt systemdienlich sein \cite{BMWKH2}. Dies zeigt, wie wichtig es ist, die Elektrolyse für zukünftige Szenarien vorzubereiten.
Um darüber hinaus das Ziel rein erneuerbarer Energien im Stromnetz zu erreichen, ist es notwendig, die Anforderungen an größere Lasten zu verändern. Dies betrifft die Systemdienstleistungen, die bisher vor allem von zentralen Großkraftwerken erbracht werden. Zukünftig sollen in Deutschland Wasserstoff-Elektrolyseanlagen in der Leistungsklasse von mehreren Megawatt aufgebaut werden, die viele Möglichkeiten bieten, das Stromnetz durch Dynamik und Regelung zu unterstützen. Daher werden in dieser Arbeit Stromrichter für die Anwendung der Wasserstoffelektrolyse untersucht, die innovative Ansätze und eine optimierte Betriebsführung ermöglichen. Anhand einer Vorauswahl werden die relevanten Topologien auf den \gls{IAF} und \gls{B6PFC} eingegrenzt, diese werden anschließend detailliert untersucht und durch Simulationsmodelle charakterisiert.\\
Zum abschließenden Vergleich der Modelle erfolgt eine Bewertung anhand des Bedarfs an induktiven und Halbleiterbauelementen sowie der Halbleiterverluste aus dem Simulationsmodell. Diese Größen werden in den einzelnen Kategorien normiert und über Gewichtungsfaktoren zu einer Gesamtbewertung zusammengefasst. Darüber hinaus wird ein Ausblick auf zukünftige Schritte wie die Optimierung der Halbleitermodelle durch Messungen und den Aufbau eines Demonstrators gegeben.
