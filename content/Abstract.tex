\addchap{Abstract}
Um in Zukunft unabhängiger von Importen zu sein und die Energieversorgung nachhaltiger zu machen, hat die Bundesregierung das Ziel zur Erzeugung von grünem Wasserstoff durch Elektrolyse im Jahr 2030 von fünf auf zehn Gigawatt Leistung erhöht. Wovon drei Gigawatt systemdienlich sein sollen \cite{BMWKH2}. Dies zeigt, wie wichtig es ist die Elektrolyse für zukünftige Szenarien vorzubereiten.
Um zusätzlich das Ziel rein erneuerbarer Energien im Stromnetz zu erreichen, wird ein Wandel in den Anforderungen an größere Lasten notwendig. Dies bezieht sich auf Systemdienstleistungen, die bisher hauptsächlich von den zentralen Großkraftwerken bereitgestellt werden. Wasserstoff-Elektrolyse-anlagen in der Leistungsklasse von mehreren Megawatt Leistung sollen in Zukunft in Deutschland aufgebaut werden, dies bietet viele Möglichkeiten durch Dynamik und Regelungen das Stromnetz zu unterstützen. Daher werden in dieser Arbeit Stromrichter für die Anwendung der Wasserstoffelektrolyse untersucht, die innovative Ansätze und eine optimierte Betriebsführung ermöglichen. Anhand einer Vorauswahl wurden die relevanten Topologien bereits auf den \gls{IAF} und \gls{B6PFC} eingegrenzt, diese werden im Detail untersucht und durch Simulationsmodelle charakterisiert.
