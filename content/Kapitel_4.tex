\chapter{Vorauswahl}
Um die möglichen Optionen einzugrenzen wird im folgenden eine Auflistung der Topologien erstellt und anhand einfacher Kriterien die Auswahl eingegrenzt. Einen guten Überblick über Schaltungen für Dreiphasige Gleichrichter mit Leistungsfaktorkorrektur gibt die Präsentation von Dominik Bortis et al. \cite{Advanced3PhPFC}. Diese bezieht sich auf Systeme mit aktiver Leistungsfaktorkorrektur. Aufgrund gewünschter Systemdienstleistungen, wie Blindleistungsbereitstellung sind Systeme mit Hybrider Kompensation nicht ausreichend.


\section{Mögliche Topologien}

Zur Eingrenzung des Lösungsraums wird zunächst eine Auflistung der möglichen Schaltungstopologien zum Anschluss an das dreiphasige Stromnetz erstellt, vgl. Tabelle \ref{tab:vorauswahl}. Die 15 aufgelisteten Topologien, begonnen mit dem in Abb. \ref{fig:B6DiodRect} dargestellten Diodengleichrichter, werden anhand der benötigten Induktivitäten, Dioden, Schalter und Stufen, sowie der Funktionsweise Hoch- bzw. Tiefstellend bewertet.\\
Tabelle \ref{tab:vorauswahl} zeigt, dass sich für eine engere Betrachtung die vier in grün hervorgehobenen Topologien eignen, da diese die im Vergleich wenigstens Induktivitäten und Halbleiter benötigen. Dioden sind aufgrund ihres simpleren Aufbaus günstiger als Leistungsschalter und fallen daher nicht so stark ins Gewicht. Die beiden anderen Topologien, 6-Switch Buck und Swiss Rectifier, werden in einer anderen Arbeit betrachtet.\\
Aufgrund der Komplexität der Schaltungen und benötigten Regelungen werden in dieser Arbeit der \gls{IAF} und \gls{B6PFC} betrachtet und die Ergebnisse für eine finale Bewertung aufbereitet.

\begin{table}
	\caption{Topologievergleich zur Vorauswahl}
	\label{tab:vorauswahl}
\begin{tabular}{|>{\centering\arraybackslash}p{3cm}|c|c|c|c|c|}
	\hline
	& Induktivitäten & Dioden & Schalter & Buck/Boost & Stufen \\
	\hline
	3-ΦDiode Bridge Rectifier & \cellcolor{yellow!25}3 &\cellcolor{red!25}6 &\cellcolor{green!25} 0 & \cellcolor{red!25}- & \cellcolor{green!25}1 \\
	\hline
	6-Switch Boost PFC Rectifier & \cellcolor{yellow!25}3 &\cellcolor{green!25} 0 & \cellcolor{green!25}6 & \cellcolor{red!25}Boost & \cellcolor{green!25}1 \\
	\hline
	Vienna Rectifier & \cellcolor{yellow!25}3 &\cellcolor{red!25}6 & \cellcolor{green!25}6 & \cellcolor{red!25}Boost & \cellcolor{green!25}1 \\
	\hline
	\cellcolor{green!10}6-Switch Buck PFC Rectifier & \cellcolor{green!25}1 &\cellcolor{red!25}6 &\cellcolor{green!25} 6 & \cellcolor{green!25}Buck & \cellcolor{green!25}1 \\
	\hline
	\cellcolor{green!10} \gls{IAF} & \cellcolor{green!25} 2 &\cellcolor{red!25}6 & \cellcolor{yellow!25}10 & \cellcolor{green!25}Buck & \cellcolor{red!25}2 \\
	\hline
	\cellcolor{green!10}Swiss Rectifier & \cellcolor{green!25}1 &\cellcolor{red!25}8 &\cellcolor{green!25} 8 & \cellcolor{green!25} Buck & \cellcolor{red!25}2 \\
	\hline
	\cellcolor{green!10} \gls{B6PFC} &\cellcolor{yellow!25}4 & \cellcolor{green!25} 0 &\cellcolor{green!25} 8 &\cellcolor{green!25} Boost/Buck & \cellcolor{red!25}2 \\
	\hline
	2/3 PWM Buck \& Boost Current Source Rectifier & \cellcolor{green!25} 1 & \cellcolor{green!25}0 & \cellcolor{yellow!25}14 & \cellcolor{green!25}Buck/Boost & \cellcolor{red!25}2 \\
	\hline
	Trident Rectifier & \cellcolor{red!25}6 &\cellcolor{green!25}0 & \cellcolor{yellow!25}12 & \cellcolor{green!25}Buck/Boost & \cellcolor{red!25}2 \\
	\hline
	Y-Rectifier & \cellcolor{yellow!25}3 &\cellcolor{green!25}0 & \cellcolor{yellow!25}12 &\cellcolor{green!25} Buck/Boost & \cellcolor{red!25}2 \\
	\hline
	3-Level Neutral Point Clamped & \cellcolor{yellow!25}3 &\cellcolor{red!25}6 & \cellcolor{yellow!25}12 & \cellcolor{red!25}Boost & \cellcolor{green!25}1 \\
	\hline
	3-Level Active Neutral Point Clamped  & \cellcolor{yellow!25}3 & \cellcolor{green!25}0 & \cellcolor{red!25}18 & \cellcolor{red!25}Boost & \cellcolor{green!25}1 \\
	\hline
	3-Level Active Neutral Point Clamped + Tiefsetzsteller & \cellcolor{yellow!25}4 &\cellcolor{green!25} 0 & \cellcolor{red!25}20 & \cellcolor{green!25}Boost/Buck & \cellcolor{red!25}2 \\
	\hline
	Three-Level Flying Capacitor (FC) Boost-Type Rectifier System & \cellcolor{yellow!25}3 & \cellcolor{green!25}0 & \cellcolor{yellow!25}12 & \cellcolor{red!25}Boost & \cellcolor{green!25}1 \\
	\hline
	Three-Level Flying Capacitor (FC + Tiefsetzsteller & \cellcolor{yellow!25}4 &\cellcolor{green!25}0 & \cellcolor{yellow!25}14 &\cellcolor{green!25} Boost/Buck &\cellcolor{red!25} 2 \\
	\hline
\end{tabular}
\end{table}
