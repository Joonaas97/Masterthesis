\chapter{Einleitung}
Um die Ziele zum Klimaschutz zu erreichen wird eine Vielzahl an Maßnahmen nötig sein, welche nur im Zusammenspiel zum Erfolg führen können. Ein großes Problem bei der Verwendung von erneuerbaren Energien ist die Volatilität dieser, daher sind deutlich größere Speichermöglichkeiten notwendig. Ein Medium zum Speichern und Transport von Energie bietet Wasserstoff, dieser kann auf verschiedene arten gewonnen werden und bietet eine Vielzahl an Einsatzmöglichkeiten. In der Industrie wird Wasserstoff bereits heute im großen Stil eingesetzt, jedoch in den meisten Fällen durch Dampfreformation aus Erdgas direkt am Einsatzort. Zukünftig kann dieser durch den Einsatz von Elektrolysezellen mit erneuerbaren Energien generiert werden. 
\cite{Elektrolyse}

\section{H2Giga}




\section{Stand der Technik}
Aktuelle Ansätze werden in \cite{HydrogenRectifier} dargestellt, diese sind wie beschrieben jedoch auf einzelne Szenarien beschränkt. Die Entwicklung der Elektrolyse läuft sehr rasant und es werden in den kommenden Jahren Änderungen erwartet, die auch die Stromversorgung betreffen. Insbesondere der Trend zu höheren Spannungsklassen ermöglicht eine deutliche Verringerung der Kosten auf Seiten der Leistungselektronik.

\section{Ziel der Arbeit}
Ziel ist es die beiden vorab ausgewählten Stromrichter Topologien anhand von detaillierten Simulationen unter gegebenen Randbedingungen zu vergleichen, um eine eindeutige Bewertung durchzuführen. Dazu werden zunächst die Randbedingungen und Eigenschaften der Schnittstellen, Elektrolyseur und Stromnetz, definiert um diese in einer Simulation mittels Matlab und der Erweiterung PLECS abzubilden. 