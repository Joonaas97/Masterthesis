\chapter{Grundlagen}


\section{Wasserstoff-Elektrolyse}

\section{Stromrichter}
\subsection{title}
\subsection{Gleichrichter}

	ungesteuerte Topologien
	
	Netzgesteuerte Topologien
	
\subsection{Power Factor Correction}
	Die \gls{PFC} ist eine nötige Maßnahme um den Blindleistungsanteil im Netz zu reduzieren 
	The front-end circuit concept of the H3R system was first introduced in late 90s by Jantsch and Verhoeve [8],
	
	
\section{IAF Rectifier}
Der \gls{IAF} Gleichrichter wurde erstmals vorgestellt in \cite{IAFfirst} im Jahr 1997 . Dieser besteht für den Hauptleistungspfad aus einem Diodengleichrichter, um passende Ströme in allen drei Phasen einzuprägen wird dieser durch ein Netzwerk aus bidirektional Sperrenden Leistungshalbleitern ergänzt, welche einen Strom in den Gleichrichter einprägen.

\section{1/3 PWM PFC Rectifier}
Bei dieser Topologie handelt es sich um eine gängige Schaltung, welche durch ein neuartiges Modulationsverfahren unter Verwendung von Induktivitäten auf der Netzseite eine Reduzierung der Schaltverluste bewirkt und Blindleistung ermöglicht. Das Verfahren wurde ausführlich von Menzi, Bortis und Kolar beschrieben \cite{1/3PWMPFC}.
\section{Bewertungskriterien}

\section{Leistungshalbleiter}

\section{Simulationssoftware}
Zur Bewertung und Betrachtung der Umsetzbarkeit, der Topologien ist es nötig diese in einer Umfassenden Simulation zu betrachten. Dies ermöglicht es die Funktionalität und den Einfluss der Parameter im direkten Zusammenspiel zu untersuchen. Insbesondere das Verhalten für Systemdienstleistungen, wie Phasenverschiebung und die dadurch beeinflusste Verteilung der Verlustleistungen. 
	\subsection{PLECS}