\begin{titlepage}
	\setlength{\parindent}{0pt}%Einrückung auf Titelseite verhindern
	\begin{figure}
		\includegraphics[height=1.4cm]{content/Grafiken/H-BRS_Logo}
		\hfill
		\includegraphics[height=2cm]{content/Grafiken/BRS}
	\end{figure}
	\vspace{1cm}
	\begin{onehalfspace}
		 Fachbereich Ingenieurwissenschaften\\
		 und Kommunikation (IWK)\\
		Studiengang Elektrotechnik M. Eng.\\
		Vertiefungsrichtung Elektronische Systementwicklung 
		\vspace{2cm}
		\begin{center}
			\begin{singlespacing}
				{\large\textsf{Master-Thesis}\par}
				\vspace{1mm}
				{\huge\textbf{\textsf{
					Netzdienliche Wasserstoff-Elektrolysegleichrichter: Eine Analyse von IAF und 1/3 PWM PFC Rectifier in der Leistungsklasse 400 kVA
				}}\par}
			\end{singlespacing}
		\end{center}
		\vfill
		Vorgelegt von:\\
		Jonas Heinemann\\
		Cecilienstraße 28\\
		53840 Troisdorf\\
		Tel. 015783841858\\
		\href{mailto:Jonas.Heinemann@h-brs.de}{Jonas.Heinemann@h-brs.de}\\
		Matr.-Nr. 9031399
		\begin{table}
			\begin{tabular}{@{}ll}
				Erstprüfer:  & Prof. Dr.-Ing. Marco Jung\\
				Zweitprüfer: & Prof. Dr. Heinrich Richard Salbert\\
			\end{tabular}
		\end{table}
	\end{onehalfspace}
	Troisdorf, den 20.01.2024
\end{titlepage}
