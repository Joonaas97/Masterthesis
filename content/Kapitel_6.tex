\chapter{Auswertung}
\section{Simulationsergebnisse}
Die Ergebnisse zur Gesamtbewertung finden sich in Tabelle \ref{tab:Auswertung}. Diese beinhaltet die Hardware, welche zum Großteil durch die Induktivitäten beeinflusst wird, sowie die Kapazitäten, Halbleiter und Treiber. Außerdem werden die Ergebnisse der Simulation durch die Verlustleistung der Halbleiter bewertet. 
\begin{table}
	\caption{Auflistung der Simulationsergebnisse und Bewertung}
	\label{tab:Auswertung}
\begin{tabular}{|c|c|c|c|c|}
	\hline
	& Topologie & B6\_Buck & IAF & Gewichtung: \\
	\hline
	Induktivitäten [uH] & L1 Netzinduktivität & 136,0 & 1,0 &  \\
	\hline
	& L2 DC Induktivität & 136,0 & 136,0 &  \\
	\hline
	& Gespeicherte Energie & 7,8 & 7,8 &  \\
	\hline
	& L3 IAF IVS Induktivität & - & 302,2 &  \\
	\hline
	& Induktivität normiert: & 1,00 & 0,43 & 50\% \\
	\hline
	Kapazitäten [uF] & C1 Netzkapazität & - & 50,0 &  \\
	\hline
	& C2 Kondensator am Elektrolyseur & 1,0 & 1,0 &  \\
	\hline
	& C3 DC Zwischenkreis & 25,0 & 50,0 &  \\
	\hline
	& Kapazität normiert: & 0,26 & 1,00 & 5\% \\
	\hline
	Halbleiter & SiC 4 mOhm & 0,0 & 2,0 &  \\
	\hline
	& SiC 2 mOhm & 10,0 & 4,0 &  \\
	\hline
	& Vienna SiC 5 mOhm & 0,0 & 6,0 &  \\
	\hline
	& SiC normiert: & 1,00 & 0,64 & 15\% \\
	\hline
	& Vienna Diode & 0,0 & 6,0 &  \\
	\hline
	& Dioden normiert & 0,0 & 1,0 & 5\% \\
	\hline
	Treiber & Treiberanzahl & 8,0 & 7,0 &  \\
	\hline
	& Treiber normiert: & 1,00 & 0,88 & 5\% \\
	\hline
	Verluste [W] & Schaltverluste 30 Grad & 567,0 & 503,0 &  \\
	\hline
	& Leitverluste 30 Grad & 254,0 & 1311,0 &  \\
	\hline
	& 30 Grad normiert: & 75\% & 75\% &  \\
	\hline
	& Schaltverluste 0 Grad & 554,0 & 511,0 &  \\
	\hline
	& Leitverluste 0 Grad & 326,0 & 748,0 &  \\
	\hline
	& 0 Grad normiert: & 25\% & 25\% &  \\
	\hline
	& Verluste normiert: & 0,50 & 1,00 & 20\% \\
	\hline
	Gesamt &  &  0,813 & 0,654 & 100\% \\
	\hline
\end{tabular}
\end{table}
\section{Auswertung}
\subsection{IAF}
Aufgrund der Anforderung an Blindleistungsbereitstellung hat die Topologie durch den \gls{IVS} einen Nachteil, da dieser sprunghafte Änderungen des Stromverlaufs verursacht. Diese starken Sprünge führen dazu, dass die \gls{THD} des Stroms deutlich verschlechtert wird. Somit kann der \gls{IAF} den Anforderungen nicht gerecht werden.