\chapter{Auswertung}
Die Ergebnisse der Gesamtbewertung sind in Tabelle \ref{tab:Auswertung} aufgeführt. Diese Tabelle enthält Informationen über die Hardware, die hauptsächlich durch die Induktivitäten beeinflusst wird, sowie über die Kapazitäten, Halbleiter und Treiber. Zusätzlich werden die Ergebnisse der Simulation anhand der Verlustleistung der Halbleiter bewertet. Für die Simulation werden die in Tabelle \ref{tab:Betriebspara} aufgeführten Betriebsparameter verwendet. Die beiden Topologien werden verglichen, um den Einfluss der Systemdienstleistungen bei einem Phasenversatz von 0° und 30° zu betrachten. Zur Durchführung eines Vergleichs der Kategorien und einer Gesamtbewertung werden die Einzelkategorien zwischen null und eins normiert und mit einem Gewichtungsfaktor summiert. Da die Drosseln einen großen Einfluss haben, werden sie mit 50 Prozent gewichtet. Die Kapazitäten haben nur einen sehr geringen Einfluss auf das Gesamtsystem und werden daher nur mit fünf Prozent bewertet. Die restlichen 45 Prozent entfallen auf die Halbleiter in Form der Chipfläche (über den RDSON), die Anzahl der Treiber und die Verlustleistung. Eine niedrigere Punktzahl führt zu einer besseren Bewertung. \\

\begin{table}
	\centering
\begin{tabular}{|c|c|}
	\hline
	Netzspannung \gls{Ull} & 617 \si{\volt} \\
	\hline
	Leistung & 200 kW bei $\varphi$ 0° \\
	\hline
	Phasenverschiebung & 0 / 30 Grad \\
	\hline
	Kühlplattentemperatur & 100 °C \\
	\hline
	Schaltfrequenz & 20 kHz \\
	\hline
\end{tabular}
\caption{Auflistung der Simulationsbetriebsparameter}
\label{tab:Betriebspara}
\end{table}


\begin{table}
\begin{tabular}{|c|c|c|c|c|}
	\hline
	& Topologie & B6\_Buck & IAF & Gewichtung: \\
	\hline
	Induktivitäten [uH] & L1 Netzinduktivität & 136,0 & 1,0 &  \\
	\hline
	 & Gespeicherte Energie [J] & 3x 4,8 & 0,1 & \\  % oder für B6 15,6 J??
	\hline
	& L2 DC Induktivität & 136,0 & 136,0 &  \\
	\hline
	& Gespeicherte Energie [J] & 7,8 & 7,8 &  \\
	\hline
	& L3 IAF IVS Induktivität & - & 302,2 &  \\
	\hline
	& Gespeicherte Energie [J] & - & 2,6 & \\
	\hline
	& L3 IAF IVS Induktivität & - & 302,2 &  \\
	\hline
	& Induktivität normiert: & 1,00 & 0,43 & 50\% \\
	\hline
	Kapazitäten [uF] & C1 Netzkapazität & - & 50,0 &  \\
	\hline
	& C2 Kondensator am Elektrolyseur & 1,0 & 1,0 &  \\
	\hline
	& C3 DC Zwischenkreis & 25,0 & 50,0 &  \\
	\hline
	& Kapazität normiert: & 0,26 & 1,00 & 5\% \\
	\hline
	Halbleiter & SiC 4 mOhm & 0,0 & 2,0 &  \\
	\hline
	& SiC 2 mOhm & 10,0 & 4,0 &  \\
	\hline
	& Vienna SiC 5 mOhm & 0,0 & 6,0 &  \\
	\hline
	& SiC normiert: & 1,00 & 0,64 & 15\% \\
	\hline
	& Vienna Diode & 0,0 & 6,0 &  \\
	\hline
	& Dioden normiert & 0,0 & 1,0 & 5\% \\
	\hline
	Treiber & Treiberanzahl & 8,0 & 7,0 &  \\
	\hline
	& Treiber normiert: & 1,00 & 0,88 & 5\% \\
	\hline
	Verluste [W] & Schaltverluste 30 Grad & 567,0 & 503,0 &  \\
	\hline
	& Leitverluste 30 Grad & 254,0 & 1311,0 &  \\
	\hline
	& 30 Grad normiert: & 75\% & 75\% &  \\
	\hline
	& Schaltverluste 0 Grad & 554,0 & 511,0 &  \\
	\hline
	& Leitverluste 0 Grad & 326,0 & 748,0 &  \\
	\hline
	& 0 Grad normiert: & 25\% & 25\% &  \\
	\hline
	& Verluste normiert: & 0,50 & 1,00 & 20\% \\
	\hline
	Gesamt &  &  0,813 & 0,654 & \\
	\hline
\end{tabular}
\caption{Auflistung der Simulationsergebnisse und Bewertung}
\label{tab:Auswertung}
\end{table}

\section{B6PFC}
	Es zeigt sich, dass der \gls{B6PFC} deutliche Nachteile bei den Induktivitäten und damit bei den Hardwarekosten hat. Die erforderliche dreiphasige Drossel führt dazu, dass der IAF in dieser Kategorie um mehr als 50\% besser abschneidet.
	Anders sieht es in den anderen Kategorien aus, wo weniger Kondensatoren benötigt werden. Die erforderliche B6-Schaltung enthält mehr \gls{MOSFET}, dafür aber keine Dioden. Bei der Verlustleistung zeigt sich der klare Vorteil der Topologie bei der Bereitstellung von \gls{SDL}, da sie fast keinen Einfluss auf die Verluste in den Halbleitern hat. Dies lässt sich anhand des Temperaturverhaltens in Abb. \ref{fig:b6temp030grad} bestätigt werden, durch die Reduzierung der Ausgangsleistung ist die Temperatur im Tiefsetzsteller etwas niedriger, rosa dargestellt. Bei den Halbleitern der B6-Brücke ist praktisch kein Unterschied zu erkennen.
	\begin{figure}
		\centering
		\includegraphics[width=0.9\linewidth]{content/Grafiken/B6_Temp_0&30Grad}
		\caption{Temperaturverhalten der Halbleiter des B6 mit (voll dargestellt) und ohne (schwach dargestellt) Phasenverschiebung}
		\label{fig:b6temp030grad}
	\end{figure}
	Die Eingangsströme sind lediglich durch die Schaltimpulse leicht verrauscht und der Sinusverlauf folgt der Eingangsspannung wie gewünscht, siehe Abbildung \ref{fig:b6acdc0grad}.  Der Stromverlauf weist eine THD von nur etwa 5,8 \% auf. 
	
	\begin{figure}[H]
		\centering
		\includegraphics[width=1\linewidth]{content/Grafiken/B6_AC+DC_0Grad}
		\caption{Eingangs- und Ausgangsgrößen ohne Phasenverschiebung}
		\label{fig:b6acdc0grad}
	\end{figure}
	Mit einer Phasenverschiebung von 30 Grad sieht das Verhalten ähnlich aus, siehe Abbildung \ref{fig:b6acdc30grad}.  Der Stromverlauf weist eine etwas höhere THD von 7,1 \% auf, die jedoch durch geeignete Filter ausgeglichen werden kann.
\begin{figure}
	\centering
	\includegraphics[width=1\linewidth]{content/Grafiken/B6_AC+DC_30Grad}
	\caption{Eingangs- und Ausgangsgrößen mit Phasenverschiebung}
	\label{fig:b6acdc30grad}
\end{figure}

\section{IAF}
Für die Auswertung werden die Simulationen für eine Dauer von 0,4 Sekunden laufen gelassen, da dann bereits ein Eingeschwungener Zustand erreicht ist. Das Temperaturverhalten der Halbleiter ist in Abb. \ref{fig:iaftemp} dargestellt, hier kann erkannt werden, dass die Kühlplattentemperatur als Startpunkt den Einschwingvorgang deutlich verkürzt. In grün dargestellt als in (a) höchste Temperatur sind die Dioden des Gleichrichters, da diese den Hauptstrom führen. Die Temperatur dieser liegt bei knapp über 140 °C und ist somit unterhalb der erlaubten maximal Temperatur. In Pink dargestellt, ist die Temperatur der T+/- Halbbrücke, diese steigt ebenfalls bei Blindleistungsbereitstellung. In Rot dargestellt und auch unabhängig von der Blindleistung ist die Temperatur des Tiefsetzstellers.  
\begin{figure}
	\centering
	\subfloat[][]{\includegraphics[width=1\linewidth]{content/Grafiken/IAF_Temp_0Grad}}
	\qquad
	\subfloat[][]{\includegraphics[width=1\linewidth]{content/Grafiken/IAF_Temp_30Grad}}
	\caption{Temperaturverhalten der Halbleiter des IAF ohne (a) und mit (b) Phasenverschiebung}
	\label{fig:iaftemp}
\end{figure}


Die Simulationsergebnisse zeigen den für die Induktivität erwarteten Strom- und Spannungsverlauf mit dreieckiger Form, siehe Abb. \ref{fig:iafacl}. Außerdem ist dem Eingangsstrom ein hochfrequenter Anteil überlagert, welcher sich durch die Schaltfrequenz des Tiefsetzstellers erklären lässt. Außerdem sieht man im Umschaltvorgang des \gls{IVS} starke Sprünge im Stromverlauf, da der Strom in der Induktivität zwischen den Phasen Kommutieren muss.\\
In Abb. \ref{fig:iafacl30grad} zeigt sich dieses Problem aufgrund der starken Spannungsunterschiede zwischen den Phasen bei Phasenverschiebung deutlich stärker. Außerdem muss der \gls{IVS} mehr Strom führen und erzeugt daher stärkere Verlustleistung. Dies ist ebenfalls an der Temperatur des in Abb. \ref{fig:iaftemp} Verlaufs zu erkennen. In (a) ist die Temperatur des IVS in grün dargestellt bei unter 110 °C und in (b) aufgrund des durch die Phasenverschiebung höheren Stroms deutlich gestiegen, bleibt aber unterhalb der zulässigen 175° für die Sperrschichttemperatur.
\begin{figure}
	\centering
	\includegraphics[width=1\linewidth]{content/Grafiken/IAF_AC+L}
	\caption[Simulationsergebnisse des IAF ohne Phasenverschiebung, Eingangsspannung und Ströme, Strom in der IVS Induktivität ]{}
	\label{fig:iafacl}
\end{figure}

\begin{figure}
	\centering
	\includegraphics[width=1\linewidth]{content/Grafiken/IAF_AC+L_30Grad}
	\caption{Simulationsergebnisse des IAF bei 30 Grad Phasenverschiebung, Eingangsspannung und Ströme, Strom in der IVS Induktivität }
	\label{fig:iafacl30grad}
\end{figure}
Aufgrund der Anforderung an Blindleistungsbereitstellung hat die Topologie durch den \gls{IVS} einen Nachteil, da dieser sprunghafte Änderungen des Stromverlaufs verursacht. Diese starken Sprünge führen dazu, dass die \gls{THD} des Stroms deutlich verschlechtert wird. Die \gls{THD} liegt bereits ohne Phasenverschiebung bei etwa 15 \% und mit verdoppelt sich diese auf etwa 32 Prozent.  Somit kann der \gls{IAF} den Anforderungen nur sehr schwer gerecht werden, da weitere Filterstufen benötigt würden. Dies lässt sich auch gut anhand der Arbeit von Schrittwieser et Al. sehen, bei deren Prototyp der Filter knapp ein viertel des Volumens ausmacht \cite{IAF99}.