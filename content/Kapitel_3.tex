\chapter{Anforderungen}
\label{chap:Anforderungen}

\section {Stromnetz}
In Deutschland sind die Vorgaben für den Anschluss von Anlagen an das Stromnetz durch den \gls{VDE} definiert. Je nach Anschlussleistung wird eine unterschiedliche Netzspannungsklasse gewählt, welche geringfügig abweichende Anschlussrichtlinien besitzt. Aufgrund der Skalierbarkeit zu höheren Leistungsklassen und der erwartbar steigenden Anforderungen, wird sich für die Bestimmungen für Hochspannung entschieden. Diese hat die Bezeichnung VDE-AR-N 4120 "Technische Regeln für den Anschluss von Kundenanlagen an das Hochspannungsnetz und deren Betrieb (TAR Hochspannung)" \cite{VDEARN4120} . 

\subsection{Systemanforderungen}


\subsection{Überspannungsschutz}

\section{Elektrolyse}

\section{Bewertungskriterien}
Die Kriterien zur finalen Auswahl der Topologie setzen sich aus der Erfüllung der Anforderungen zusammen sowie der Bewertung der Hardware. Die Grundlegenden Anforderungen aus Seiten des Stromnetzes und Elektrolyseur wurden bereits in der Vorauswahl berücksichtigt und können nun im Detail anhand von \gls{THD} und Rippelgrößen betrachtet werden. Die Quantifizierung der Hardware wird zum einen anhand der Verlustleistung in den Halbleitern, welche indirekt auch den Kühlungsaufwand repliziert, zum anderen durch die Größe und den Aufwand für die Komponenten berücksichtigt.  